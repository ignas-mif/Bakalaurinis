\documentclass{VUMIFPSkursinis}
\usepackage{algorithmicx}
\usepackage{algorithm}
\usepackage{algpseudocode}
\usepackage{amsfonts}
\usepackage{amsmath}
\usepackage{bm}
\usepackage{caption}
\usepackage{color}
\usepackage{float}
\usepackage{graphicx}
\usepackage{listings}
\usepackage{subfig}
\usepackage{wrapfig}

% Titulinio aprašas
\university{Vilniaus universitetas}
\faculty{Matematikos ir informatikos institutas}
\department{Programų sistemų katedra}
\papertype{Kursinis darbas}
\title{Prekės pardavimo prognozavimas iš vaizdo ir aprašymo naudojant giliuosius neuroninius tinklus}
\titleineng{Predicting advert sale from its image and description using deep learning networks}
\status{4 kurso 5 grupės studentas}
\author{Ignas Bradauskas}
\supervisor{Linas Petkevičius, J. Asist.}
\date{Vilnius – \the\year}

% Nustatymai
%\setmainfont{Palemonas}   % Pakeisti teksto šriftą į Palemonas (turi būti įdiegtas sistemoje)
\bibliography{bibliografija}

\begin{document}
\maketitle

\tableofcontents

\sectionnonum{Įvadas}


Amerikos įmonė „Starbucks“ 2007-ais metais patyrė staigų pelno kritimą (nuo 1 bilijono dolerių iki 504 milijonų). 
Buvo būtina imtis veiksmų ir įmonės valdžia nusprendė pasitelkti naujausias technologijas spręsti šiai problemai. Jie sukūrė lojalumo programą, dovanų kortelių programą ir, turbūt svarbiausia, elektroninę parduotuvę „store.starbucks.com“. \cite{ElCom}
Šis ankstyvus modernizavimasis lėmė įstaigos sėkmę ir leido jai tapti viena garsiausių kavos įmonių pasaulyje. Kiti
pasaulio mažmeninės prekybos verslai irgi neatsilieka ir pradeda naudoti interneto teigiamas paslaugas. Naujausi statistiniai duomenys
teigia, jog elektoroninė komercija (e-komercija) sudaro apie 9,5\% JAV mažmeninės prekybos rinkos dalies ir prognozuojama, jog ši dalis toliau sparčiai augs. Nuo ketivtojo 2017-tųjų ketvirčio iki pirmojo 2018-tųjų ji išaugo \textit{3,9\%  (±0.7\%)}.  \cite{US}  Iš duomenų matoma, jog prekyba internetu, šiais laikais, yra kaip niekad aktuali ir sparčiai auganti. Dėl šios priežasties, šiame darbe bus kuriamas dar daug dėmesio technologijų srityje nesulaukęs GTTE-DNT gilusis modelis.

Modelis spręs prekės pardavimo greičio prognozės uždavinį. Didelę prekės pardavimo proceso dalį užima rinkos analizė. Darbe aprašomas DNT būtų pajėgus atlikti rinkos analizės dalį užvartotoją - kuo trumpesnis laiko tarpas iki prognazuojamo pardavimo, tuo prekė patrauklesnė. Būtų galima nesudėtingai šią metriką paversti į kokią kitą, labiau atitinkančią žmogaus kompiuterio sąveikos geriausias praktikas.

Šiai problemai planuojama naudoti įrankius leisiančius apdoroti duomenis nuotraukas, universalų sakinių užkoduotoją (angl. universal sentance encoder, toliau USE), skirtą apdoroti prekės aprašymą bei pilnai sujungtą neuroninį tinklą. Tinkle turėtų būti optimizuojami parametrai gamma pasiskirstymo funkcijai, siekant sukurti lygtį, lengvai analizuojamą statistinių metodų. DNT apsimokinti panaudotų istorinius skelbimų duomenis. Programa apdorotų šią informaciją ir jos galutinė išvestis būtų tikimybinė prognozė, kada objektas bus parduotas. Tokį rezultatą būtų patogu pritaikyti spręsti prieš tai minėtą uždavinį. Sistema taip pat galėtų paimti visus aktyvius skelbimus, juos surikiuoti pagal prognozuojamą pardavimo datą arba, kitais žodžiais, pagal jų patrauklumą ir pradiniame puslapyje pavaizduoti numatomai sėkmingiausius.

\textbf{Darbo tikslas} - suformuluoti prekių pardavimo laiko progozavimo uždavinį, kuriuo remiantis bus sukurta ir išbandyta prototipinė sistema naudojanti giliuosius neuroninius tinklus.

\vspace{3mm} %5mm vertical space

\textbf{Darbo uždaviniai:}
\begin{enumerate}
  \item Atlikti DNT apžvalgą.% svarbu palyginti su kitomis sistemomis 
  \item Sudaryti netikrų duomenų rinkinį.
  \item Suformuluoti prekių pardavimo prognozavimo uždavinį GN tinklams.
  \item Sukurti prototipą ir atlikti tikslumo vertinimą.
\end{enumerate}

\textbf{Laukiami rezultatai}: Tyrimo proceso eigoje bus praktiškai pritaikomos žinios ir kuriamas DNT, kuris, bus pajėgus atlikti prekės pardavimo prognozę su 85\% (±5\%)tikslumu. Taip pat bus analizuojama aktuali uždavinui medžiaga susidedanti iš: dirbtinių neuroninių tinklų veikimo principų, universalaus sakinių užkoduotojo, iš anksto apmokinto ResNet tinklo pritaikymo kitam uždaviniui, gamma skirstinio, uždaviniui aktualių statistikos skyrių. Atsižvelgiant į darbo kūrimo sėkmingumą ir spartą, bus priimamas sprendimas ar į darbą įtraukti cenzūruotų duomenų pritaikymą, t.y. duomenų apie įvykio nebuvimą panaudojimą tinklo apmokinimui.

\textbf{Tyrimo metodas}: Šis tyrimas bus paremtas kiekybiniu metodu, nes bus daromas sisteminis tyrimas, kurio metu bus tiriamas GTTE-DNT tinklo sugeneruotos prognozės procentalus atitikimas realiems duomenims.

\textbf{Numatomas darbo atlikimo procesas}: Darbo atlikimo procesas bus iteratyvus, t.y. nebus apibrėžiami visi sistemos reikalavimai pačioje aplikacijos kūrimo proceso pradžioje. Vietoje to,sistema bus statoma ciklais, mažomis užduotimis. Taip siekiama sumažinti riziką ir optimizuoti rezultatus.

Kiekviename cikle bus išsikeliami tikslai, gilinimasi į šaltinius, realizuojama praktinė programos dalis arba rašto darbo dalis ir kursinio vadovui pristatomi rezultatai.

\textbf{Darbui aktualūs literatūriniai šaltiniai}: Didelė dalis šios medžiagos analizės remsis Egil Martinsson „WTTE-RNN : Weibull Time To Event Recurrent Neural Network“ magistro darbu,  kuriame buvo sprendžiama sudėtingesnė ir bendresnė prognozės uždavinio forma remiantis Weibull skirstinio parametrų optimizacija.

Taip pat bus pasitelkiami kiti šaltiniai aiškinantys atitinkamas temas, t.y. apmokinimo perdavimui bus naudojami Stanford cs231 užrašai, siekiant našiau dirbti ir skirti mažiau laiko žemo  lygio kodui bus naudojamas TensorFlow karkasas ir jo dokumentacija. USE analizei bus naudojamas Google mokslininkų parašytas Universal Sentence Encoder 2018 mokslinis darbas. Gamma skirstiniui aprašyti bus remimasi Thomas P. Minka „Estimating a Gamma distribution“. Kiti šaltiniai dar nėra numatyti arba yra per mažai jais remimasi, kad būtų vertinga juos minėti šioje skiltyje.

Darbo autorius naudos keras karkasą kurti modeliui, taip pat bus naudojamos python beautiful soup ir TorRequest bibliotekos kurti interneto gremžėją \textit{(angl. web scrapper)} skirtą sudaryti duomenų rinkinį. 


\section{Metodologinė dalis}
% Medžiagos darbo tema dėstymo skyriuose pateikiamos nagrinėjamos temos detalės:
% pradinė medžiaga, jos analizės ir apdorojimo metodai, sprendimų įgyvendinimas,
% gautų rezultatų apibendrinimas. Šios dalies turinys labai priklauso nuo darbo
% temos. Skyriai gali turėti poskyrius ir smulkesnes sudėtines dalis, kaip
% punktus ir papunkčius.

% Medžiaga turi būti dėstoma aiškiai, pateikiant argumentus. Tekstas dėstomas
% trečiuoju asmeniu, t.y. rašoma ne „aš manau“, bet „autorius mano“, „autoriaus
% nuomone“. Reikėtų vengti informacijos nesuteikiančių frazių, pvz., „...kaip jau
% buvo minėta...“, „...kaip visiems žinoma...“ ir pan., vengti grožinės literatūros
% ar publicistinio stiliaus, gausių metaforų ar panašių meninės išraiškos
% priemonių.

Siekiant geriau suprasti esamų sistemų atliekančių prekių pardavimo analizę veikimo principus, šiame skyriuje bus analizuojamos DNT sąvokos ir architektūriniai modeliai. Kadangi, prototipams rašyti buvo naudojamas aukšto lygio API, bus stengiamasi nesileisti į smulkias implementacijos detales ir formalius įrodymus, bet siekiant sukurti darbą suprantamą skaityotojams neturintiems patirties dirbtinio intelekto sferoje, bus aptariama plati DNT sąvokų ir įrankių aibė.

\subsection{Į priekį sklindančių neuroninų tinklų analizė}

\subsection{Konvoliucinių (sąsukinių) neuroninių tinklų analizė}

\subsection{ResNet analizė}

\subsection{Žinių perdavimas}

\subsection{USE analiz}


\subsection{WTTE-RNN analizė}

Šiame skyriuje bus atliekama WTTE-RNN modelio analizė. Skyriaus pradžioje bus pamenami ir trumpai apibūdinami pagrindiniai tinklui suprasti reikalingi mechanizmai ir konceptai. Poskyriuose bus detaliau analizuojamos visos sąvokos. 

Šio modelio pagrindinis tikslas yra analizuoti visus objekto įvykius iki lemtingojo įvykio užbaigiančio tiriamo objekto aktualumą (mirtis, pardavimas, sugedimas) ir pagal juos sugeneruoti skirstinį. Ji yra sudaroma pagal formulę, kuriai reikia dviejų parametrų \textit{alfa} ir \textit{beta}. Nuo šių dviejų parametrų priklauso skirstinio forma. Tai galima tarti, jog DNT turi sugeneruoti šiuos du parametrus. Gero tinklo rezultatas turėtų būti skirstinys turintis didžiają dalį ploto arčiau dabarties, būtent tiems objektams, kuriems labai greitai atsitiks lemtingas įvykis. O tiems, kurie nesusidurs su lemtingu įvykiu dar ilgai, turėtų būti sugeneruojamas labai ploksčias skirstinys. Ją turint apskaičiuoti prognozę yra itin nesudėtinga.

% Praktikos dalyje papasakoti kad aprašytas wtte naudojamas duomenų rinkinys netobulai sueina su manuoju.
Besigilinant į šį modelį didžiaja dalimi bus remimasi Egil Martinsson magistriniu darbu „WTTE-RNN : Weibull Time To Event Recurrent Neural Network“. Šis darbas daugiau dėmesio skiria \textit{nelemiamiems} įvykiams, t.y. rekurentiniams. Toki įvykiai nesustabdo tiriamo objekto gyvavimo, jie gali nuolatos atsitikti jo egzistavimo metu. Įvykiai sustabdantys objekto egzistavimą laikomi išskirtiniais. Turint tai omenyje, galime teigti, kad tinklas tinkamas tokiems uždaviniams kaip: mirties laiko prognozė, prietaiso gedimo laiko prognozė ir objekto pardavimo prognozė. Pastarasis uždavinys bus detaliau aptariamas tolimesniuose darbo skyriuose.

Minėtieji rekurentiniai įvykiai yra svarbus informacijos šaltinis WTTE-RNN tinklui, jie taip pat suteikia galimybę apmokinimui panaudoti duomenis, kurie vadimani cenzūruotais. Toki duomenys išreiškia, kad įvykis dar neįvykęs, bet mes galime būti tikri, jog jis buvo neįvykęs bent iki dabar. Geras to pavyzdys būtų žmogaus prekių pirkimo duomenys. Tai yra, mes nežinome kiek laiko praeis iki kol vartotojas nusipirks naują prekę, bet mes žinome, kad bus praėję bent tiek laiko kiek praėjo nuo vartotojo praeitos prekės nusipirkimo iki dabar. WTTE-RNN tinklo autorius argumentuoja, jog duomenys apie įvykio nebuvimą yra svarbi bendro duomenų rinkinio dalis. Pasak jo, tinklas pasieks daug geresnius rezultatus jeigu bus apmokinamas cenzūruotais ir necenzūruotais duomenimis. Žemiau pavaizduotame paveiksliuke parodytas pavyzdys abiejų rūšių duomenų (žr. x pav).

\begin{figure}[H]
  \centering
  \includegraphics[scale=0.8]{img/censored}
  \caption{ Mes turime įvykius pažymėtus $v${\scriptsize t}. Prieš paskutinį užfiksuotą įvykį mes turime \textit{tikruosius} laikus iki įvykio, tad \~{$y$}{\scriptsize $t$} = $y${\scriptsize $t$}. Po to, mes turime cenzūruotą laiką iki įvykio \~{$y$}{\scriptsize $t$} $\leq$ {$y$}{\scriptsize $t$}\cite{WTTEBLOG}.}
  \label{img:censored}
\end{figure}

Atsižvelgiant į tai ar objekto duomenys yra cenzūruoti ar necenzūruoti tinklas turi naudoti skirtingas skirtingas \textit{loss} funkcijas. Šios funkcijos leidžia sukurti specialiai pritaikytus skirstinius dviems skirtingiems duomenų tipams. Pavyzdžiui, necenzūruotiems duomenims sukurtas skirstinys turėtų turėti daugiausiai svorio prie įvykio (žr. pav. x), necenzūruoti duomenys turi sugeneruoti skirstinį daugiausiai ploto užimantį už užcenzūravimo taško (žr. pav. x), o necenzūruotų ir cenzūruotų duomenų mišinys turi sukurti kompromisuojantį, labiau išbalansuotą skirstinį (žr. pav. x).  

\begin{figure}[H]
  \centering
  \includegraphics[scale=0.8]{img/pdf1}
  \caption{ Necenzūruotas stebėjimas\cite{WTTEBLOG}.}
  \label{img:pdf1}
\end{figure}


\begin{figure}[H]
  \centering
  \includegraphics[scale=0.8]{img/pdf2}
  \caption{ Cenzūruotas stebėjimas\cite{WTTEBLOG}.}
  \label{img:pdf2}
\end{figure}


\begin{figure}[H]
  \centering
  \includegraphics[scale=0.8]{img/pdf3}
  \caption{ Cenzūruotas ir necenzūruotas stebėjimas \cite{WTTEBLOG}.}
  \label{img:pdf3}
\end{figure}

Paveikslėliuose matomos formulės yra \textit{loss} formulės gaunamos atsižvelgiant į kokio tipo duomenys gauti. 

%To be continued..? 


\subsection{Skirstiniai}

Atsižvelgiant į tą faktą, jog pagrindinis darbo DNT yra stipriai susijęs su skirstiniais, yra svarbu atlikti šio statistikos mokslo įrankio pagrindų apžvalgą. 

Skirstiniai apibūdina dydžių pasiskirstymo dėsnį. Tai reiškia, kad jie parodo visas įmanomas duomenų rinkinio reikšmes ir kaip dažnai tos reikšmės kartojasi. Turint visa tai omenyje, nebūtų klaidinga tarti, jog skirstinys pavaizduoja duomenų rinkinio formą ir, dažnai, turint duomenų rinkinį yra siekiama pagal jį sugeneruoti kuo tinkamesnį skirstinį. Tai yra trokštama, nes pagal jį galima išvesti didelį kiekį išvadų apie duomenis, pavyzdžiui, padaryti išvadą kurią dieną kokia tikimybė parduoti objektą.

Skirstinių generavimo formulės paprastai turi keletą parametrų skirtų nustatyti jo formą. Šie pasiskirstymo dėsniai taip pat turi keletą svarbių aptarti sąvokų. Aktualiausia darbui iš jų yra \textit{kvantilis}, bet pradėkime ne nuo jo, nes jį paaiškinti taps elementariau žinant apie Culmulative Distribution Function (toliau CDF). 

CDF funkcija apskaičiuoja tikimybę, kad kintamasis įgaus reikšmę mažesnę arba lygią x. Tai yra, jeigu skirstinys atspindi žmonių ugių pasiskirstymą, tai CDF apskaičiuotų kokia tikimybė, jog atsitiktinis žmogus bus mažesnio arba lygaus x ūgiui. 

Čia paveiksliukas.

Kvantilis yra apskaičiuojamas su Present Point Function (toliau PPF). Ši funkcija yra invertuota CDF funkcija, todėl ji dažnai literatūroje apibūdinama kaip invertuota distribucijos funkcija. Tokia būsena reiškia, jog funkcija pradeda nuo tikimybės ir, turint ją, apskaičiuoja x. Prisimenant prieš tai aptartą ūgių pavyzdį, jį galime pritaikyti šiai funkcijai. Tai yra, jeigu skirstinys atspindi žmonių ūgių pasiskirstymą, tai PPF funkcija galėtų paskaičiuos iki kokio ūgio yra x\% procentų žmonių. Ši sąvoka bus naudojama tolimesniuose skyriuose skaičiuoti per kiek dienų objektas bus parduotas su x\% tikimybe.

Čia paveiksliukas.

Svarbu paminėti, kad yra sukurti kelių tipų skirstiniai. Galima atspėti, kad originaliame Egil Martinsson straipsnyje, kuriuo stipriai remimasi šiame darbe, buvo remimasi Weibull skirstiniu. Pasak WTTE-RNN autoriaus, šis skirstinys taip pat pasižymi kitomis savybėmis padarančiomis jį itin tinkamu:

\begin{enumerate}
  \item Lengvai diskretizuojamas.
  \item Unimodalus bet išraiškingas.
  \item Vietos-Skalės transformacija.
  \item Regularizacijos mechanizmai.
  \item Kt.
\end{enumerate}

Weibull skirstinys dažnai naudojamas išgyvenimo analizėje ir patikimumo inžinerijoje, bet jis nevienintelis. Kiti šiose srityse dažnai naudojami skirstiniai yra: exponentinis, Log-logistinis, Exponentinis-logaritminis, gamma, Rayleigh, Erlang. Tinklui aprašomam kituose skyriuose galima pritaikyti ne tiktai Weibull skirstinį.

\section{Tiriamoji dalis}

Siekiant susidaryti tinkamą WTTE tinklo gerumo įvertinimą, šalia jo, buvo sukurtas paprastesnis DNT. Šis DNT neatsižvelgia į cenzūruotus duomenis ir nekuria skirstinio, visais kitais atžvilgiais jis yra panašus į pagrindinį tyrimo objektą - WTTE tinklą.

Tiriamojoje dalyje bus peržvelgiami abiejų sukurtųjų tinklų modeliai, treniravimo greičiai, tikslumai. Bus pateikiami sugeneruotų duomenų grafikai, skirstiniai. Taip pat, bus aptariamas duomenų surinkimo ir pritaikymo tinklų apmokinimo reikmėms procesas bei sunkumai.


\subsection{Siūlomas modelis}



\subsection{Duomenų rinkinio sudarymas}

\subsubsection{Duomenų gavimas}

\subsubsection{Duomenų apdorojimas}

\subsection{Modelio kūrimas}

\sectionnonum{Rezultatai ir išvados}

% Nors sukurti deramo prototipo spręsti įvardintam uždaviniui, grubiai tariant, nepavyko, darbas atvėrė galimybę stipriam žinių ir patirties pamatui susiformuoti. Dėl šio pamato tolimesnieji autoriaus tyrimai šioje srityje tūrėtų būti sėkmingesni.

% Rezultatų ir išvadų dalyje turi būti aiškiai išdėstomi pagrindiniai darbo
% rezultatai (kažkas išanalizuota, kažkas sukurta, kažkas įdiegta) ir pateikiamos
% išvados (daromi nagrinėtų problemų sprendimo metodų palyginimai, teikiamos
% rekomendacijos, akcentuojamos naujovės).
\printbibliography[heading=bibintoc]  % Šaltinių sąraše nurodoma panaudota
% literatūra, kitokie šaltiniai. Abėcėlės tvarka išdėstomi darbe panaudotų
% (cituotų, perfrazuotų ar bent paminėtų) mokslo leidinių, kitokių publikacijų
% bibliografiniai aprašai.  Šaltinių sąrašas spausdinamas iš naujo puslapio.
% Aprašai pateikiami netransliteruoti. Šaltinių sąraše negali būti tokių
% šaltinių, kurie nebuvo paminėti tekste.

% \sectionnonum{Sąvokų apibrėžimai}
\sectionnonum{Santrumpos}


\begin{enumerate}
  \item DNT - dirbtinis neuroninis tinklas \textit{(angl. artificial neural network)} tai tarpusavyje sujungtų mazgų grupė primenanti smegenų neuronų tinklą.
  \item CNN - konvoliucinis neuroninis tinklas \textit{(angl. convolutional neural network)} tai neuroninių tinklų klasė, dažniausiai naudojama vaizdų analizei.
  \item ResNet - giliusis liekaninis tinklas \textit{(angl. residual neural network)} tai DNT išsiskiriantis naudojamais sujungimais skirtais praleisti sluoksnius.
  \item FNN - į priekį sklendantis neuroninis tinklas \textit{(angl. feedforward neural network)} tai neuroninis tinklas kurio kievienas mazgas, yra sujungtas su prieš tai esančio sluoksnio kiekvienu mazgu. Jame mazgų jungtys nesuformuoja ciklo, informacija sklinda tiktai į priekį.
  \item Išeigos požymių žemėlapis \textit{(angl. output feature map)} - tai konvoliucinio sluoksnio filtro rastų požymių rinkinys.
  \item Konvoliucinis sluoksnis \textit{(angl. convolutional layer)} - tai vienas iš CNN sluoksnių.
  \item API \textit{(application programing interface)} - rinkinys protokolų, subrutininių definicijų ir įrankių skirtų kurti programinę įrangą.
  \item GDPR \textit{(angl. general data protection regulation)} - duomenų privatumo apsaugos aktas taikomas kiekvienam žmogui Europos Sąjungoje.
  \item SVM \textit{(angl. support vector machine)} - mokinimo su mokytoju DNT modelis, kuris naudoja paraminius vektorius \textit{(angl. support vectors)}, o ne sluoksnių metodą.
  \item TOR \textit{(angl. the onion router)} - anoniminį naršymą įgalinantis protokolas gebantis dinamiškai keisti IP adresą.
  \item DDOS \textit{(angl. denial of service attack)} - kibernetinės atakos būdas peremtas dideliu duomenų srautu.
  \item GN - gilusis neuroninis.
\end{enumerate}
% Sąvokų apibrėžimai ir santrumpų sąrašas sudaromas tada, kai darbo tekste
% vartojami specialūs paaiškinimo reikalaujantys terminai ir rečiau sutinkamos
% santrumpos.

\appendix  % Priedai
% Prieduose gali būti pateikiama pagalbinė, ypač darbo autoriaus savarankiškai
% parengta, medžiaga. Savarankiški priedai gali būti pateikiami ir
% kompaktiniame diske. Priedai taip pat numeruojami ir vadinami. Darbo tekstas
% su priedais susiejamas nuorodomis.
\section{Sukurtų aplikacijų ir duomenų rinkinio repozitorija}

https://github.com/ignas-mif/Kursinis2018

\end{document}
